\section{Nozioni generali sull'utilizzo di HEC-HMS}
\subsection{Metodo CN-SCS}
Il metodo CN-SCS permette di quantificare il volume di pioggia che può diventare deflusso di runoff, a seconda delle caratteristiche del suolo e del soprassuolo.\\
Infatti, un suolo impermeabilizzato risponderà in modo differente rispetto ad uno vegetato.\\
Semplificando, è possibile quantificare le tendenze di permeabilità del suolo e del soprassuolo andando a quantificarle mediante una scala da 0 a 100 (ovvero dalla condizione più permeabile alla più impermeabile).\\
Generalmente, i valori di CN (\textit{curve number}) sono riportati in varie tabelle, a seconda della natura del bacino di studio.\\
Per esempio, al seguente link si riportano le tabelle rese disponibili dal manuale di utilizzo del manuale di HEC-HMS: \cite{cn_tables_hec_hms}.

\subsection{Time Lag}
Il tempo di lag (ritardo) è lo scostamento che interessa una qualsiasi quantità d'acqua in movimento all'interno di un elemento del bacino (un sottobacino, un tratto di canale,...).\\
Questo metodo non genera alcuna perdita di volume di acqua o diffusione del processo. 