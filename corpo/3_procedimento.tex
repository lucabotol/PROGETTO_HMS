\section{Analisi dei dati}
In questa sezione si andrà ad esporre la procedura di analisi di una qualsiasi serie pluviometrica. Per semplificare la lettura della relazione, i risultati numerici verranno riportati nei capitoli successivi.\\
Mediante l'elaborazione dei dati pluviometrici è possibile mettere in relazione l'entità degli eventi estremi e la loro probabilità di accadimento, che in ogni caso è in relazione inversamente proporzionale. Questo processo, successivamente, permetterà di estrarre dal campione di dati (riferiti ad alcuni anni), la LSPP (Linea Segnalatrice di Probabilità Pluviometrica).\\
Per questa relazione, i dati pluviometrici ci sono stati forniti dal Professore, pur essendo di un'area diversa da quella di studio.
\subsection{POT}
Generalmente, il campione della serie è formato dai massimi annuali di precipitazione di tutta la popolazione.\\
In questo caso invece, i valori verranno estratti secondo il metodo dei \textit{Peak over Threshold} (Picchi superiori al limite, POT): fissata una soglia inferiore, si andrà a considerare solamente i valori uguali o superiori ad essa.\\
Matematicamente, il metodo POT viene descritto così:

   \begin{equation} 
  P(Y_{\text{max}} < x) = \sum_{n=1}^{\infty} p(n) F(x)^n = \sum_{n=1}^{\infty} \frac{\lambda^n e^{-n}}{n!} \left\{ 1 - \left[ 1 + \frac{\xi}{\psi} (x - s) \right]^{\frac{1}{\xi}} \right\}^n
\label{equazione_pot}   
\end{equation}
All'interno dell'equazione \ref{equazione_pot} sono presenti due formule di distribuzione:
\begin{itemize}
    \item Distribuzione di Poisson: rappresenta la distribuzione di probabilità discreta dell'evento di studio;
    \item Distribuzione generalizzata di Pareto: rappresenta la distribuzione dei valori eccedenti alla soglia di riferimento posta.
\end{itemize}

Il parametro di soglia (\textit{s}) viene posto a discrezione da chi esegue i calcoli; generalmente, il numero di valori da isolare dovrebbe coincidere con il numero di anni della serie.\\
Il parametro $\xi$ si ricava dalla formula:
\begin{equation}
    \xi = \frac{1}{2} \left[1- \left( \frac{\mu - s}{\sigma} \right)^2 \right] 
\end{equation}

Il valore di $\psi$ si calcola mediante la formula:
\begin{equation}
    \psi = \sigma(1-\xi) \sqrt{(1-2 \cdot \xi)}
\end{equation}

Il parametro $\lambda$ si ricava dal rapporto tra il numero di valori considerati e gli anni della serie:
\begin{equation}
    \lambda = \frac{n}{N}
\end{equation}

Infine, la probabilità di superamento dell'evento $F_{x_i}$ si ricava dalla distribuzione generalizzata di Pareto (GPD):
\begin{equation}
F_{x_i} = 1- \left[ 1+\xi\left(\frac{x_i - s}{\psi}\right)\right]^{-\frac{1}{\xi}}
\label{prob_non_super_pareto}
\end{equation}

Il vantaggio di questo metodo, rispetto a quello che interessa solamente i singoli massimi annui, è che il campione risultante considera anche gli eventi di entità minori.

\subsection{Analisi preliminare di elaborazione dati}
Per poter estrarre l'altezza critica di precipitazione, data una serie pluviometrica, è necessario svolgere alcune procedure preliminari sulla popolazione:
\begin{itemize}
    \item ordinare in ordine decrescente i valori di precipitazione;
    \item fissare una soglia inferiore di riferimento;
    \item eliminare i valori di precipitazione minori, solamente nel caso in cui ce ne fossero alcuni dipendenti da altri;
    \item procedere con l'elaborazione statistica.
\end{itemize}

\subsection{Plotting position}
Mediante la plotting position è possibile verificare la bontà della distribuzione dei valori.\\
Ad ogni valore della serie, posta in ordine crescente, viene attribuito un valore dipendente alla sua posizione numerica, tramite la formula:
\begin{equation}
   P= \frac{i}{n+1}
\end{equation}
Questo parametro indica la possibilità di non superamento dell'evento, che sarà sempre maggiore all'aumentare della posizione nella serie.

\subsection{Valutazione visiva dell'accoppiamento}
Avendo calcolato la plotting position per ogni valore della serie, è possibile associare la relativa probabilità di non superamento dell'evento, attraverso la formula \ref{prob_non_super_pareto}.\\
Infine, per effettuare la valutazione visiva della bontà di accoppiamento, è necessario creare un grafico dove in ascissa viene posta la plotting position ed in ordinata la probabilità di non superamento dell'evento.


\subsection{Altezza critica di precipitazione}
Dopo aver processato i dati del campione statistico ed aver valutato la bontà di accoppiamento con la serie teorica, è possibile calcolare la relativa altezza critica di precipitazione (dato un certo tempo di ritorno), mediante la formula:
\begin{equation}
    h_{T_R} = s + \frac{\psi}{\xi} \left[ \left(\frac{1}{\lambda \cdot T_R} \right) ^{-\xi} - 1 \right]
    \label{h_critica_tr}
\end{equation}

\subsection{LSPP}
La Linea Segnalatrice di Possibilità Pluviometrica è una funzione che mette in relazione la durata di un certo evento pluviometrico e l'altezza di pioggia prevista, per un dato tempo di ritorno.\\
Quindi è possibile, svolgendo i calcoli per diversi tempi di ritorno, avere diverse LSPP per una stessa stazione di rilevamento (o per uno stesso sito).\\
Al fine di semplificare la lettura del grafico, ed eventualmente l'estrapolazione del dato di pioggia, è possibile imporre che gli assi cartesiani abbiano la scala logaritmica; in questo modo, la linea curva interpolatrice dei punti diventa una retta.\\
La traiettoria della linea segue l'andamento della funzione $h=at^n$, dove:
\begin{itemize}
\item $t$ è il parametro della durata dell'evento (espresso in ore);
\item $a$ dipende dal tempo di ritorno;    
\item $n$ è un valore compreso tra 0 e 1, ed è praticamente costante per una data stazione.
\end{itemize}

\subsection{Calcolo del tempo di corrivazione}
Il tempo di corrivazione $t_c$ di un bacino è il periodo necessario affinché una qualsiasi massa d'acqua percorra lo spazio tra la sezione di chiusura ed il punto più idraulicamente distante ad essa.\\
Questo tempo può anche essere visto come il periodo di pioggia necessario affinché ogni area del bacino sia contribuente al deflusso alla sezione di chiusura.\\
Questo parametro è molto importante in idrologia, perché alla durata di precipitazione uguale a quella di corrivazione avviene il massimo deflusso. Se la precipitazione avesse un tempo inferiore, non tutto il bacino sarebbe contribuente al medesimo istante; se la durata di pioggia fosse superiore a quella di corrivazione, l'intensità sarebbe minore.\\
Esistono diverse formule per calcolare il $t_c$ di un bacino, in questa relazione ho scelto di usare quella di Ferro (del 2002):
\begin{equation}
    T_c = 0.675 \cdot \sqrt{A}
\end{equation}
Nella formula occorre introdurre il valore di estensione in chilometri quadri, al fine di ottenere il tempo in ore.\\
Svolgendo la formula, si ottiene che il tempo di corrivazione per il bacino del Boite è pari a 13.21 h, e quello del Piave è pari a 19.24 h.