\section{Conclusioni}
In questa relazione si è andati a studiare, tramite un modello idrologico, le risposte di due bacini montani, a fronte di una precipitazione con tempo di ritorno pari a 215 anni.\\
Mediante analisi statistico-probabilistica è stata ricavata la curva LSPP per il medesimo periodo di ritorno; tale funzione è pari a $h=61.517 \cdot t ^{0.512}$.\\
Dalla funzione interpolatrice sono state ricavate le altezze di pioggia da introdurre nel software HEC-HMS, in modo da permettergli di generale il diagramma degli afflussi-deflussi, mediante un modello già calibrato in precedenza.\\
A fronte di una precipitazione con tempo di ritorno pari a 215 anni, il bacino del Boite genera un picco di deflusso di 458.7 $\frac{m^3}{s}$, mentre il bacino del Piave ne genera uno complessivo di 446.5 $\frac{m^3}{s}$.\\
L'altro risultato importante da considerare è il volume defluito alla sezione di chiusura alla fine dell'evento pluviometrico, che è pari a 101.49 $mm$ per il bacino del Boite e 43.53 $mm$ per il bacino del Piave.