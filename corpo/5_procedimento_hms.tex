\section{Procedimento su HEC-HMS}
Dopo aver ottenuto la funzione della LSPP per il proprio tempo di ritorno, è possibile studiare l'analisi idrologica su HEC-HMS.\\
Per fare ciò, è necessario creare un modello idrologico per ogni bacino, in modo da immettere in input i valori pluviometrici calcolati precedentemente.\\
Il \textit{modello idrologico} è una semplificazione del fenomeno reale (nel nostro caso la trasformazione degli afflussi in deflussi).\\
Il modello idrologico, affinché possa essere soggetto della simulazione, deve avere i seguenti parametri:
\begin{itemize}
    \item bacino idrologico: indica la struttura del bacino, comprendente gli eventuali sottobacini o particolari elementi idraulici;
    \item modello meteorologico: sintetizza le caratteristiche pluviometriche dell'evento meteorologico, come per esempio i bacini interessati o gli effetti ambientali del vento e dell'evapotraspirazione;
    \item specifiche di controllo: possiedono le caratteristiche generali della simulazione da effettuare;
    \item serie di dati temporali: rappresentano i valori (noti) che è possibile introdurre nel modello meteorologico, come per esempio valori pluviometrici o di deflusso.
\end{itemize}
 Ulteriori proprietà, come per esempio il modello digitale del terreno, possono essere introdotti successivamente nel software.\\
 Come anticipato precedentemente, in questa relazione si andrà a studiare la risposta idrologica di due bacini, quindi risulta necessario creare due modelli di calcolo.

 \subsection{Preparazione del modello idrologico}
 Al fine di cominciare l'analisi in HEC-HMS, è necessario introdurre nel software i parametri precedentemente elencati:
 \begin{itemize}
    \item bacino idrologico: sono stati introdotti i file raster dei bacini (\ref{bacino_boite} e \ref{bacino_piave}) e valori di primo tentativo di CN e Lag time di piena;
    \item modello meteorologico: 
 \end{itemize}
 
 \subsection{Calibrare il modello idrologico}
 