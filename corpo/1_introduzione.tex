\section{Introduzione}
In questa relazione si andrà ad esporre i procedimenti necessari per svolgere l'analisi di risposta idrologica di due bacini montani, mediante l'analisi statistico-probabilistica delle piogge e l'utilizzo di un software di modellazione idrologica (HEC-HMS).\\
Infatti, la prima parte della relazione interesserà l'analisi statistico-probabilistica delle serie storiche pluviometriche, in modo da ottenere la corretta curva LSPP per un dato tempo di ritorno.\\
Successivamente, utilizzando i dati di pioggia e di deflusso misurati durante l'evento Vaia, sarà possibile creare ed ottimizzare il modello idrologico dei due bacini di studio.\\
Infine, dal risultato dell'analisi statistico-probabilistica, si riuscirà ad ottenere le risposte dei bacini a fronte di una precipitazione con un qualsiasi tempo di ritorno. 